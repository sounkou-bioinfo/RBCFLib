\documentclass[12pt]{article}
\batchmode
\usepackage{hyperref}

\topmargin -0.5in
\textheight 9.0in
\oddsidemargin 0pt
\evensidemargin 0pt
\textwidth 6.5in

%-------------------------------------------------------------------------------
% get epsf.tex file, for encapsulated postscript files:
\input epsf
%-------------------------------------------------------------------------------
% macro for Postscript figures the easy way
% usage:  \postscript{file.ps}{scale}
% where scale is 1.0 for 100%, 0.5 for 50% reduction, etc.
%
\newcommand{\postscript}[2]
{\setlength{\epsfxsize}{#2\hsize}
\centerline{\epsfbox{#1}}}
%-------------------------------------------------------------------------------

\title{User's Guide for SuiteSparseQR, a multifrontal multithreaded sparse
QR factorization package (with optional GPU acceleration)}
\author{Timothy A. Davis\thanks{
email: DrTimothyAldenDavis@gmail.com.
http://www.suitesparse.com.
Portions of this work were supported by the National
Science Foundation, under grants 0203270, 0620286, and 0619080.},
Sencer Nuri Yeralan, Sanjay Ranka, Wissam Sid-Lakhdar}

% version of SuiteSparse/SPQR
\date{VERSION 4.3.2, Jan 20, 2024}


%-------------------------------------------------------------------------------
\begin{document}
%-------------------------------------------------------------------------------
\maketitle

\begin{abstract}

SuiteSparseQR is an implementation of the multifrontal sparse QR factorization
method.  Parallelism is exploited both in the BLAS and across different frontal
matrices using Intel's Threading Building Blocks, a shared-memory programming
model for modern multicore architectures.  It can obtain a substantial fraction
of the theoretical peak performance of a multicore computer.  The package is
written in C++ with user interfaces for MATLAB, C, and C++.  Both real and
complex sparse matrices are supported.

\end{abstract}

\maketitle

%-------------------------------------------------------------------------------
\section{Introduction}
\label{intro}
%-------------------------------------------------------------------------------

The algorithms used in SuiteSparseQR are discussed in a companion paper,
\cite{Davis08a}, and an overview of how to use the software is given in
\cite{Davis08b}.  This document gives detailed information on the installation
and use of SuiteSparseQR.

SPQR, Copyright (c) 2008-2023, Timothy A Davis. All Rights Reserved.

SPDX-License-Identifier: GPL-2.0+

The GPU modules in SPQRGPU are under a different copyright:

    SPQRGPU, Copyright (c) 2008-2023, Timothy A Davis, Sanjay Ranka,
    Sencer Nuri Yeralan, and Wissam Sid-Lakhdar, All Rights Reserved.

%-------------------------------------------------------------------------------
\section{Using SuiteSparseQR in MATLAB}
%-------------------------------------------------------------------------------

The simplest way to use SuiteSparseQR is via MATLAB.  Its syntax includes every
feature of the MATLAB \verb'qr' in version R2009a and earlier
\cite{GilbertMolerSchreiber}, plus additional features not available in MATLAB.
It is also a replacement for \verb'x=A\b' for least-squares problems and
underdetermined systems.  In addition to substantial gains in performance (10x
to 100x is not uncommon, up to 10,000x has been observed), SuiteSparseQR adds
new capabilities that are not present in MATLAB.  For example, it provides an
efficient method for finding the minimum 2-norm solution to an underdetermined
system.

%-------------------------------------------------------------------------------
\subsection{Installing SuiteSparseQR for use in MATLAB}
%-------------------------------------------------------------------------------

All packages in SuiteSparse, including SuiteSparseQR and the codes it relies on
(AMD, COLAMD, CHOLMOD, METIS, CCAMD, and CCOLAMD) are compiled with a single
command typed into the MATLAB Command Window.  SuiteSparseQR uses the LAPACK
and BLAS libraries provided with MATLAB; you do not need to do anything to use
these.  Below are step-by-step instructions for compiling all of SuiteSparse
(including SuiteSparseQR), and optional instructions on using METIS.

%-------------------------------------------------------------------------------
\subsubsection{Now you're ready to compile (on any operating system)}
%-------------------------------------------------------------------------------

Type these commands in the MATLAB window:
\begin{verbatim}
    cd SuiteSparse
    SuiteSparse_install
\end{verbatim}
You will be asked if you want to run some demos.  I recommend that you do this
to ensure your functions have been installed correctly.  Next type the command
\begin{verbatim}
    pathtool
\end{verbatim}
and examine your MATLAB path.  The various SuiteSparse directories have been
placed in your path.  Click ``save'' to save this path for future MATLAB
sessions.  If this fails, you do not have permission to modify the
\verb'pathdef.m' file (it is shared by all users).

SuiteSparse now has a \verb'SuiteSparse_paths.m' that you can add to
your \verb'startup.m'.
An alternative is to type the command:
\begin{verbatim}
    path
\end{verbatim}
and cut-and-paste the paths displayed there into your own \verb'startup.m'
file, prepending the command \verb'addpath' to each line.

Your \verb'startup.m' file should appear in the directory in which MATLAB
starts.  Failing that, every time you start MATLAB, find your \verb'startup.m'
file and run it.  For more help, type \verb'doc startup' in MATLAB.

The \verb'SuiteSparse_install' script works on any version of MATLAB
(Linux/Unix, Mac, or Windows) if you have a C++ compiler.  The install script
will detect if you have placed the METIS directory in the right place, and will
compile it for use with SuiteSparseQR if it finds it there.  Otherwise METIS
will be skipped (the install script will tell you if it finds METIS or not).

%-------------------------------------------------------------------------------
\subsection{Functions provided to the MATLAB user}
%-------------------------------------------------------------------------------

Three primary functions are available:

\begin{enumerate}

\item \verb'spqr', a replacement for the MATLAB \verb'qr'

\item \verb'spqr_solve', a replacement for \verb'x=A\b' when \verb'A' is sparse
and rectangular.  It works for the square case, too, but \verb'x=A\b' will be
faster (using LU or Cholesky factorization). \verb'spqr_solve' is a good method
for ill-conditioned or rank-deficient square matrices, however.

\item \verb'spqr_qmult', which multiplies \verb'Q' (stored in Householder
vector form) times a matrix \verb'x'.

\end{enumerate}

Their syntax is described below in the table below.
The permutation \verb'P' is chosen to reduce fill-in and to return \verb'R' in
upper trapezoidal form if \verb'A' is estimated to have less than full rank.
The \verb'opts' parameter provides non-default options (refer to the next
section).  The output \verb'Q' can be optionally returned in Householder form,
which is far sparser than returning \verb'Q' as a sparse matrix.

\vspace{0.1in}

{\footnotesize
\begin{tabular}{|ll|}
\hline
\verb'R = spqr (A)'             & Q-less QR factorization \\
\verb'R = spqr (A,0)'           & economy variant (\verb'size(R,1) = min(m,n)') \\
\verb'R = spqr (A,opts)'        & as above, with non-default options \\
\hline
\verb'[Q,R] = spqr (A)'         & \verb'A=Q*R' factorization \\
\verb'[Q,R] = spqr (A,0)'       & economy variant (\verb'size(Q,2) = size(R,1) = min(m,n)') \\
\verb'[Q,R] = spqr (A,opts)'    & \verb'A=Q*R', with non-default options \\
\hline
\verb'[Q,R,P] = spqr (A)'       & \verb'A*P=Q*R' where P reduces fill-in \\
\verb'[Q,R,P] = spqr (A,0)'     & economy variant (\verb'size(Q,2) = size(R,1) = min(m,n)') \\
\verb'[Q,R,P] = spqr (A,opts)'  & as above, with non-default options \\
\hline
\verb'[C,R] = spqr (A,B)'       & as \verb'R=spqr(A)', also returns \verb"C=Q'*B" \\
\verb'[C,R] = spqr (A,B,0)'     & economy variant (\verb'size(C,1) = size(R,1) = min(m,n)') \\
\verb'[C,R] = spqr (A,B,opts)'  & as above, with non-default options \\
\hline
\verb'[C,R,P] = spqr (A,B)'     & as \verb'R=spqr(A*P)', also returns \verb"C=Q'*B" \\
\verb'[C,R,P] = spqr (A,B,0)'   & economy variant (\verb'size(C,1) = size(R,1) = min(m,n)') \\
\verb'[C,R,P] = spqr (A,B,opts)'& as above, with non-default options \\
\hline
\verb'x = spqr_solve (A,B)'     & \verb'x=A\B' \\
\verb'[x,info] = spqr_solve (A,B,opts)' & as above, with statistics and non-default parameters \\
\hline
\verb'Y = spqr_qmult (Q,X,k)'   & computes \verb"Q'*X", \verb"Q*X", \verb"X*Q'", or \verb"X*Q"
(selected with \verb'k') \\
\hline
\end{tabular}
}

%-------------------------------------------------------------------------------
\subsection{The {\tt opts} parameter}
%-------------------------------------------------------------------------------

The \verb'opts' struct provides control over non-default parameters for
SuiteSparseQR.  Entries not present in \verb'opts' are set to their defaults.

\begin{itemize}

    \item \verb'opts.tol':   columns that have 2-norm \verb'<= opts.tol' are
    treated as zero. The default is \verb"20*(m+n)*eps*sqrt(max(diag(A'*A)))"
    where \verb'[m n]=size(A)'.

    \item \verb'opts.econ':  number of rows of \verb'R' and columns of \verb'Q'
    to return.  The default is \verb'm'.  Using \verb'n' gives the standard
    economy form (as in the MATLAB \verb'qr(A,0)').  A value less than the
    estimated rank \verb'r' is set to \verb'r', so \verb'opts.econ=0' gives
    the ``rank-sized'' factorization, where \verb'size(R,1)==nnz(diag(R))==r'.

    \item \verb'opts.ordering': a string describing which column ordering
    method to use.  Let \newline
    \verb'[m2 n2]=size(S)' where \verb'S' is obtained by
    removing singletons from \verb'A'.  The singleton permutation places
    \verb'A*P' in the form \verb'[A11 A12 ; 0 S]' where \verb'A11' is upper
    triangular with diagonal entries all greater than \verb'opts.tol'.

    The default is to use COLAMD if \verb'm2<=2*n2'; otherwise try AMD.  Let
    \verb'f' be the flops for \verb"chol((S*P)'*(S*P))" with the ordering
    \verb'P' found by AMD.  Then if \verb'f/nnz(R) >= 500' and
    \verb'nnz(R)/nnz(S) >= 5' then try METIS, and take the best ordering found
    (AMD or METIS); otherwise use AMD without trying METIS.  If METIS is not
    installed then the default ordering is to use COLAMD if \verb'm2<=2*n2' and
    to use AMD otherwise.

    The available orderings are:

       {\tt 'default'}: the default ordering.

       {\tt 'amd'}: use \verb"amd(S'*S)".

       {\tt 'colamd'}: use \verb"colamd(S)".

       {\tt 'metis'}: use \verb"metis(S'*S)", only if METIS is
       installed.

       {\tt 'best'}: try all three (AMD, COLAMD, METIS) and take the
       best.

       {\tt 'bestamd'}: try AMD and COLAMD and take the best.

       {\tt 'fixed'}: use \verb"P=I"; this is the only option if
       \verb"P" is not present in the output.

       {\tt 'natural'}: singleton removal only.

    \item \verb'opts.Q': a string describing how \verb'Q' is to be returned.
    The default is \verb"'discard'" if \verb'Q' is not present in the output,
    or \verb"'matrix'" otherwise.  If \verb'Q' is present and \verb'opts.Q' is
    \verb"'discard'", then \verb"Q=[]" is returned (thus \verb"R=spqr(A*P)" is
    \verb"[Q,R,P]=spqr(A)" where \verb"spqr" finds \verb"P" but \verb"Q" is
    discarded instead). The usage \verb"opts.Q='matrix'" returns \verb'Q' as a
    sparse matrix where \verb'A=Q*R' or \verb"A*P=Q*R".  Using
    \verb"opts.Q='Householder'" returns \verb'Q' as a struct containing the
    Householder reflections applied to \verb'A' to obtain \verb'R', resulting
    in a far sparser \verb'Q' than the \verb"'matrix'" option.

    \item \verb'opts.permutation': a string describing how \verb'P' is to be
    returned.  The default is \verb"'matrix'", so that \verb"A*P=Q*R".  Using
    \verb"'vector'" gives \verb"A(:,P)=Q*R" instead.

    \item \verb'opts.spumoni': an integer \verb'k' that
        acts just like \verb"spparms('spumoni',k)".

    \item \verb'opts.min2norm': used by \verb'spqr_solve'; you can use
    \verb"'basic'" (the default), or \verb"'min2norm'".  Determines the kind of
    solution that \verb'spqr_solve' computes for underdetermined systems.  Has
    no effect for least-squares problems; ignored by \verb'spqr' itself.

\end{itemize}

%-------------------------------------------------------------------------------
\subsection{Examples on how to use the MATLAB interface}
%-------------------------------------------------------------------------------

To solve a least-squares problem, or to find the basic solution to an
underdetermined system, just use \verb'x = spqr_solve(A,b)' in place of
\verb'x=A\b'.  To compute the QR factorization, use \verb'[Q,R]=spqr(A)'
instead of \verb'[Q,R]=qr(A)'.  Better results can be obtained by discarding
\verb'Q' with the usage \verb'R=spqr(A)' (in place of \verb'R=qr(A)'), or by
requesting \verb'Q' in Householder form with \verb'[Q,R]=spqr(A,opts)' where
\verb"opts.Q='Householder'".  The latter option is not available in MATLAB.  To
use a fill-reducing ordering, simply use any of the syntaxes above with
\verb'P' as an output parameter.

The least-squares solution of an overdetermined system \verb'A*x=b' with
\verb'm>n' (where \verb'A' has rank \verb'n') can be found in one of at least
seven ways (in increasing order of efficiency, in time and memory):

{\footnotesize
\begin{tabular}{|l|l|}
    \hline
    \verb"x = pinv(full(A)) * b ;"
        & impossible for large \verb'A' \\
    \hline
    \verb"[Q,R] = spqr (A) ;"
        & high fill-in in \verb'R', \\
    \verb"x = R\(Q'*b) ;"
        & \verb'Q' costly in matrix form \\
    \hline
    \verb"[Q,R,P] = spqr (A) ;"
        & low fill-in in \verb'R', \\
    \verb"x = P*(R\(Q'*b)) ;"
        & \verb'Q' costly in matrix form \\
    \hline
    \verb"[Q,R,P] = spqr (A,struct('Q','Householder')) ;"
        & low fill-in in \verb'R', \\
    \verb"x = P*(R\spqr_qmult (Q,b,0)) ;"
        & \verb'Q' in efficient Householder form \\
    \hline
    \verb"[c,R,P] = spqr (A,b) ;"
        & \verb'Q' not kept, \\
    \verb"x = P*(R\c) ;"
        & \verb'P' a permutation matrix \\
    \hline
    \verb"[c,R,p] = spqr (A,b,0) ;"
        & \verb'Q' not kept, \\
    \verb"y = (R\c) ; x(p) = y"
        & \verb'p' a permutation vector \\
    \hline
    \verb"x = spqr_solve (A,b) ;"
        & less memory and better handling \\
        & of rank-deficient matrices \\
    \hline
\end{tabular}
}

\vspace{0.1in}

The minimum-norm solution of an underdetermined system \verb'A*x=b' with
\verb'm<n' can be found in one of five ways (in increasing order of
efficiency, in time and memory):

\vspace{0.1in}

{\footnotesize
\begin{tabular}{|l|l|}
    \hline
    \verb"x = pinv(full(A)) * b ;"
        & impossible for large \verb'A' \\
    \hline
    \verb"[Q,R] = spqr (A') ;"
        & high fill-in in \verb'R', \\
    \verb"x = Q*(R'\b) ;"
        & \verb'Q' costly in matrix form \\
    \hline
    \verb"[Q,R,P] = spqr (A') ;"
        & low fill-in in \verb'R', \\
    \verb"x = Q*(R'\(P'*b)) ;"
        & \verb'Q' costly in matrix form \\
    \hline
    \verb"[Q,R,P] = spqr (A',struct('Q','Householder')) ;"
        & low fill-in in \verb'R', \\
    \verb"x = spqr_qmult (Q,R'\(P'*b),1) ;"
        & \verb'Q' in efficient Householder form \\
    \hline
    \verb"opts.solution = 'min2norm' ;"
        & as 4th option above, but faster, \\
    \verb"x = spqr_solve (A,b,opts) ;"
        & less memory, and better handling \\
        & of rank-deficient matrices \\
    \hline
\end{tabular}
}

Note that \verb'spqr_solve' uses a fill-reducing ordering, by default.
It can be disabled or modified using a non-default \verb'opts' parameter
(\verb'opts.ordering', specifically).

%-------------------------------------------------------------------------------
\section{Using SuiteSparseQR in C and C++}
%-------------------------------------------------------------------------------

SuiteSparseQR relies on CHOLMOD for its basic sparse matrix data structure, a
compressed sparse column format.  CHOLMOD provides interfaces to the AMD,
COLAMD, and METIS ordering methods, supernodal symbolic Cholesky factorization
(namely, \verb'symbfact' in MATLAB), functions for converting between different
data structures, and for basic operations such as transpose, matrix multiply,
reading a matrix from a file, writing a matrix to a file, and many other
functions.

%-------------------------------------------------------------------------------
\subsection{Installing the C/C++ library}
%-------------------------------------------------------------------------------

In Linux/MacOs, type \verb'make' at
the command line, in either the \verb'SuiteSparse' directory (which
compiles all of SuiteSparse) or in the \verb'SuiteSparse/SPQR' directory (which
just compiles SuiteSparseQR and the libraries it requires).  SuiteSparseQR will
be compiled, and a set of simple demos will be run (including the one in the
next section).

The use of \verb'make' is optional.  The top-level \verb'SPQR/Makefile' is
a simple wrapper that uses \verb'cmake' to do the actual build.
The \verb'CMakeLists.txt' file can be imported into MS Visual Studio,
for example.

If SuiteSparseQR is compiled with \verb'-DNEXPERT', the ``expert'' routines in
\verb'SuiteSparseQR_expert.cpp' are not compiled.  The expert routines are
included by default.

To fully test 100\% of the lines of SuiteSparseQR, go to the \verb'Tcov'
directory and type \verb'make'.  This will work for Linux only.

To install the shared library
into /usr/local/lib and /usr/local/include, do {\tt make install}.
To uninstall, do {\tt make uninstall}.
For more options, see the {\tt SuiteSparse/README.txt} file.

%-------------------------------------------------------------------------------
\subsection{C/C++ Example}
%-------------------------------------------------------------------------------

The C++ interface is written using templates for handling both real and complex
matrices.  The simplest function computes the MATLAB equivalent of
\verb'x=A\b' and is almost as simple:
{\footnotesize
\begin{verbatim}
    #include "SuiteSparseQR.hpp"
    X = SuiteSparseQR <double> (A, B, cc) ;
\end{verbatim}
}
The C version of this function is almost identical:
{\footnotesize
\begin{verbatim}
    #include "SuiteSparseQR_C.h"
    X = SuiteSparseQR_C_backslash_default (A, B, cc) ;
\end{verbatim}
}

Below is a simple C++ program that illustrates the use of SuiteSparseQR
(with 64-bit integer indices).  The
program reads in a least-squares problem from \verb'stdin' in MatrixMarket
format \cite{BoisvertPozoRemingtonBarrettDongarra97}, solves it, and prints the
norm of the residual and the estimated rank of \verb'A'.  The comments reflect
the MATLAB equivalent statements.  The C version of this program is identical
except for the \verb'#include' statement and call to SuiteSparseQR which are
replaced with the C version of the statement above, and C-style comments.

{\footnotesize
\begin{verbatim}
    #include "SuiteSparseQR.hpp"
    int main (int argc, char **argv)
    {
        cholmod_common Common, *cc ;
        cholmod_sparse *A ;
        cholmod_dense *X, *B, *Residual ;
        double rnorm, one [2] = {1,0}, minusone [2] = {-1,0} ;
        int mtype ;

        // start CHOLMOD
        cc = &Common ;
        cholmod_l_start (cc) ;

        // load A
        A = (cholmod_sparse *) cholmod_l_read_matrix (stdin, 1, &mtype, cc) ;

        // B = ones (size (A,1),1)
        B = cholmod_l_ones (A->nrow, 1, A->xtype, cc) ;

        // X = A\B
        X = SuiteSparseQR <double> (A, B, cc) ;

        // rnorm = norm (B-A*X)
        Residual = cholmod_l_copy_dense (B, cc) ;
        cholmod_l_sdmult (A, 0, minusone, one, X, Residual, cc) ;
        rnorm = cholmod_l_norm_dense (Residual, 2, cc) ;
        printf ("2-norm of residual: %8.1e\n", rnorm) ;
        printf ("rank %ld\n", cc->SPQR_istat [4]) ;

        // free everything and finish CHOLMOD
        cholmod_l_free_dense (&Residual, cc) ;
        cholmod_l_free_sparse (&A, cc) ;
        cholmod_l_free_dense (&X, cc) ;
        cholmod_l_free_dense (&B, cc) ;
        cholmod_l_finish (cc) ;
        return (0) ;
    }
\end{verbatim}
}

To use SuiteSparseQR with 32-bit integer indices in all of its matrices,
simply replace all \verb'cholmod_l_*' calls abovce to \verb'cholmod_*',
and use this for SPQR:

{\footnotesize
\begin{verbatim}
        X = SuiteSparseQR <double,int32_t> (A, B, cc) ;
\end{verbatim}
}

%-------------------------------------------------------------------------------
\subsection{C++ Syntax}
%-------------------------------------------------------------------------------

All features available to the MATLAB user are also available to both the C and
C++ interfaces using a syntax that is not much more complicated than the MATLAB
syntax.  Additional features not available via the MATLAB interface include the
ability to compute the symbolic and numeric factorizations separately (for
multiple matrices with the same nonzero pattern but different numerical
values).  The following is a list of user-callable C++ functions and what they
can do:

\begin{enumerate}

    \item \verb'SuiteSparseQR': an overloaded function that provides functions
    equivalent to \verb'spqr' and \verb'spqr_solve' in the SuiteSparseQR MATLAB
    interface.

    \item \verb'SuiteSparseQR_factorize': performs both the symbolic and
    numeric factorizations and returns a QR factorization object such that
    \verb'A*P=Q*R'.  It always exploits singletons.

    \item \verb'SuiteSparseQR_symbolic': performs the symbolic factorization
    and returns a QR factorization object to be passed to
    \verb'SuiteSparseQR_numeric'.  It does not exploit singletons.

    \item \verb'SuiteSparseQR_numeric': performs the numeric factorization on a
    QR factorization object, either one constructed by
    \verb'SuiteSparseQR_symbolic', or reusing one from a prior call to
    \verb'SuiteSparseQR_numeric' for a matrix \verb'A' with the same pattern as
    the first one, but with different numerical values.

    \item \verb'SuiteSparseQR_solve': solves a linear system using the object
    returned by \newline \verb'SuiteSparseQR_factorize' or
    \verb'SuiteSparseQR_numeric', namely \verb"x=R\b", \newline \verb"x=P*R\b",
    \verb"x=R'\b", or \verb"x=R'\(P'*b)".

    \item \verb'SuiteSparseQR_qmult': provides the same function as
    \verb'spqr_qmult' in the MATLAB interface, computing
    \verb"y=Q'*x", \verb"y=Q*x", \verb"y=x*Q'", or \verb"y=x*Q".
    It uses the efficient Householder reprensentation of \verb'Q', which
    represents a square orthonormal matrix computed by
    \verb'SuiteSparseQR_factorize' or \verb'SuiteSparseQR_numeric'.  The
    Householder representation always represents a square orthonormal matrix,
    regardless of whether \verb'Q' is a full or economy factor.
    \verb'SuiteSparseQR_qmult' applies this square matrix to compute its result
    \verb'y'.

    \item \verb'SuiteSparseQR_min2norm': finds the minimum 2-norm solution to
    an underdetermined linear system.

    \item \verb'SuiteSparseQR_free': frees the QR factorization object.

\end{enumerate}

%-------------------------------------------------------------------------------
\subsection{Details of the C/C++ Syntax}
%-------------------------------------------------------------------------------

For further details of how to use the C/C++ syntax, please refer to the
definitions and descriptions in the following files:

\begin{enumerate}
\item \verb'SuiteSparse/SPQR/Include/SuiteSparseQR.hpp' describes each
C++ function.  Both \verb'double' and \verb'std::complex<double>' matrices
are supported.

\item \verb'SuiteSparse/SPQR/Include/SuiteSparseQR_definitions.h' describes
definitions \newline common to both C and C++ functions.  For example, each of
the ordering methods is given a \verb'#define''d name.  The default is
\verb'ordering = SPQR_ORDERING_DEFAULT', and the default tolerance is given by
\verb'tol = SPQR_DEFAULT_TOL'.

\item \verb'SuiteSparse/SPQR/Include/SuiteSparseQR_C.h' describes
the C-callable functions.

\end{enumerate}

Version 4.0 of SuiteSparseQR adds a 32-bit version, where the indices of its
sparse matrices are all \verb'int32_t', contributed by Raye Kimmerer.  To use
this version, add \verb'int32_t' as the second template parameter to all C++
methods, and use the \verb'cholmod_*' methods instead of \verb'cholmod_l_*'
to create and access its input/output matrices.

The C/C++ options corresponding to the MATLAB \verb'opts' parameters and the
contents of the optional \verb'info' output of \verb'spqr_solve' are described
below.  Let \verb'cc' be the CHOLMOD \verb'Common' object, containing parameter
settings and statistics.  All are of type \verb'double', except for
\verb'SPQR_istat' which is \verb'int64_t',
\verb'cc->memory_usage' which is
\verb'size_t', and \verb'cc->SPQR_nthreads' which is \verb'int'.  Parameters
include:

\vspace{0.1in}
{\footnotesize
\begin{tabular}{|ll|}
\hline
\verb'cc->SPQR_grain' & the same as \verb'opts.grain' in the MATLAB interface \\
\verb'cc->SPQR_small' & the same as \verb'opts.small' in the MATLAB interface \\
\verb'cc->SPQR_nthreads'
    & the same as \verb'opts.nthreads' in the MATLAB interface \\
\hline
\end{tabular}
}
\vspace{0.1in}

Other parameters, such as \verb'opts.ordering' and \verb'opts.tol',
are input parameters to the various C/C++ functions.  Others such as
\verb"opts.solution='min2norm'" are separate functions in the C/C++
interface.  Refer to the files listed above for details.
Output statistics include:

\vspace{0.1in}
{\footnotesize
\begin{tabular}{|ll|}
\hline
\verb'cc->SPQR_flopcount_bound' & an upper bound on the flop count \\
\verb'cc->SPQR_tol_used' & the tolerance used (\verb'opts.tol') \\
\hline
\verb'cc->SPQR_istat [0]' & upper bound on \verb'nnz(R)' \\
\verb'cc->SPQR_istat [1]' & upper bound on \verb'nnz(H)' \\
\verb'cc->SPQR_istat [2]' & number of frontal matrices \\
\verb'cc->SPQR_istat [3]' & unused \\
\verb'cc->SPQR_istat [4]' & estimate of the rank of \verb'A' \\
\verb'cc->SPQR_istat [5]' & number of column singletons \\
\verb'cc->SPQR_istat [6]' & number of row singletons \\
\verb'cc->SPQR_istat [7]' & ordering used \\
\hline
\verb'cc->memory_usage'   & memory used, in bytes \\
\hline
\end{tabular}
}
\vspace{0.1in}

The upper bound on the flop count is found in the analysis phase, which ignores
the numerical values of \verb'A' (the same analysis phase operates on both real
and complex matrices).  Thus, if you are factorizing a complex matrix, multiply
this statistic by 4.

%-------------------------------------------------------------------------------
\section{GPU acceleration}
\label{GPU}
%-------------------------------------------------------------------------------

As of version 2.0.0, SuiteSparseQR now includes GPU acceleration.
It can exploit a single NVIDIA GPU, via CUDA.
The packages SuiteSparse\_GPURuntime and
GPUQREngine are also required (they should appear in the SuiteSparse
directory, along with SPQR).

At run time, you must also enable the GPU by setting \verb'Common->useGPU'
to \verb'true'.  Before calling any SuiteSparseQR function, you must
poll the GPU to set the available memory.  Below is a sample code
that initializes CHOLMOD and then polls the GPU for use in SuiteSparseQR.

\begin{verbatim}
    size_t total_mem, available_mem ;
    cholmod_common *cc, Common ;
    cc = &Common ;
    cholmod_l_start (cc) ;
    cc->useGPU = true ;
    cholmod_l_gpu_memorysize (&total_mem, &available_mem, cc) ;
    cc->gpuMemorySize = available_mem ;
    if (cc->gpuMemorySize <= 1)
    {
        printf ("no GPU available\n") ;
    }

    // Subsequent calls to SuiteSparseQR will use the GPU, if available
\end{verbatim}

See \verb'Demo/qrdemo_gpu.cpp' for an extended example, which can be
compiled via \verb'make gpu' in the \verb'Demo' directory.

GPU acceleration is not yet available via the MATLAB mexFunction interface.
We expect to include this in a future release.

For a detailed technical report on the GPU-accelerated algorithm,
see \verb'qrgpu_paper.pdf' in the \verb'Doc' directory.

%-------------------------------------------------------------------------------
\section{Requirements and Availability}
\label{summary}
%-------------------------------------------------------------------------------

SuiteSparseQR requires four prior Collected Algorithms of the ACM: CHOLMOD
\cite{ChenDavisHagerRajamanickam09,DavisHager09} (version 1.7 or later), AMD
\cite{AmestoyDavisDuff96,AmestoyDavisDuff03}, and COLAMD
\cite{DavisGilbertLarimoreNg00_algo,DavisGilbertLarimoreNg00} for its
ordering/analysis phase and for its basic sparse matrix data structure, and the
BLAS \cite{dddh:90} for dense matrix computations on its frontal matrices; also
required is LAPACK \cite{LAPACK} for its Householder reflections.  An efficient
implementation of the BLAS is strongly recommended, either vendor-provided
(such as the Intel MKL, the AMD ACML, or the Sun Performance Library) or other
high-performance BLAS such as those of \cite{GotoVanDeGeijn08}.

The use of Intel's Threading Building Blocks is optional \cite{Reinders07}, but
without it, only parallelism within the BLAS can be exploited (if available).
Suite\-SparseQR can optionally use METIS 4.0.1 \cite{KarypisKumar98e} and two
constrained minimum degree ordering algorithms, CCOLAMD and CAMD
\cite{ChenDavisHagerRajamanickam09}, for its fill-reducing ordering options.
SuiteSparseQR can be compiled without these ordering methods.

In addition to appearing as Collected Algorithm 8xx of the ACM, SuiteSparseQR
is available at
\url{http://www.suitesparse.com}.
See SPQR/Doc/License.txt for the license.
Alternative licenses are also
available; contact the author for details.

%-------------------------------------------------------------------------------
% References
%-------------------------------------------------------------------------------

\bibliographystyle{plain}
\bibliography{spqr_user_guide}
\end{document}

